\documentclass[12pt,]{article}
\usepackage{lmodern}
\usepackage{amssymb,amsmath}
\usepackage{ifxetex,ifluatex}
\usepackage{fixltx2e} % provides \textsubscript
\ifnum 0\ifxetex 1\fi\ifluatex 1\fi=0 % if pdftex
  \usepackage[T1]{fontenc}
  \usepackage[utf8]{inputenc}
\else % if luatex or xelatex
  \ifxetex
    \usepackage{mathspec}
    \usepackage{xltxtra,xunicode}
  \else
    \usepackage{fontspec}
  \fi
  \defaultfontfeatures{Mapping=tex-text,Scale=MatchLowercase}
  \newcommand{\euro}{€}
    \setmainfont{Georgia}
\fi
% use upquote if available, for straight quotes in verbatim environments
\IfFileExists{upquote.sty}{\usepackage{upquote}}{}
% use microtype if available
\IfFileExists{microtype.sty}{%
\usepackage{microtype}
\UseMicrotypeSet[protrusion]{basicmath} % disable protrusion for tt fonts
}{}
\usepackage[margin=1in]{geometry}
\ifxetex
  \usepackage[setpagesize=false, % page size defined by xetex
              unicode=false, % unicode breaks when used with xetex
              xetex]{hyperref}
\else
  \usepackage[unicode=true]{hyperref}
\fi
\hypersetup{breaklinks=true,
            bookmarks=true,
            pdfauthor={},
            pdftitle={},
            colorlinks=true,
            citecolor=blue,
            urlcolor=blue,
            linkcolor=magenta,
            pdfborder={0 0 0}}
\urlstyle{same}  % don't use monospace font for urls
\setlength{\parindent}{0pt}
\setlength{\parskip}{6pt plus 2pt minus 1pt}
\setlength{\emergencystretch}{3em}  % prevent overfull lines
\providecommand{\tightlist}{%
  \setlength{\itemsep}{0pt}\setlength{\parskip}{0pt}}
\setcounter{secnumdepth}{0}

%%% Use protect on footnotes to avoid problems with footnotes in titles
\let\rmarkdownfootnote\footnote%
\def\footnote{\protect\rmarkdownfootnote}

%%% Change title format to be more compact
\usepackage{titling}

% Create subtitle command for use in maketitle
\newcommand{\subtitle}[1]{
  \posttitle{
    \begin{center}\large#1\end{center}
    }
}

\setlength{\droptitle}{-2em}
  \title{}
  \pretitle{\vspace{\droptitle}}
  \posttitle{}
  \author{}
  \preauthor{}\postauthor{}
  \date{}
  \predate{}\postdate{}

\usepackage{booktabs}
\usepackage[final]{changes}
\usepackage[font={small},labelfont=bf,labelsep=colon]{caption}
\linespread{2}
\usepackage[compact]{titlesec}
\usepackage{enumitem}
\usepackage{tikz}
\def\checkmark{\tikz\fill[scale=0.4](0,.35) -- (.25,0) -- (1,.7) -- (.25,.15) -- cycle;}
\setlist{nolistsep}
\titlespacing{\section}{2pt}{*0}{*0}
\titlespacing{\subsection}{2pt}{*0}{*0}
\titlespacing{\subsubsection}{2pt}{*0}{*0}
\setlength{\parskip}{3pt}
\setremarkmarkup{(#2)}

% Redefines (sub)paragraphs to behave more like sections
\ifx\paragraph\undefined\else
\let\oldparagraph\paragraph
\renewcommand{\paragraph}[1]{\oldparagraph{#1}\mbox{}}
\fi
\ifx\subparagraph\undefined\else
\let\oldsubparagraph\subparagraph
\renewcommand{\subparagraph}[1]{\oldsubparagraph{#1}\mbox{}}
\fi

\begin{document}
\maketitle

\section{Figure Captions}\label{figure-captions}

\textbf{Figure 1:} Workflow illustration for the proposed pipeline.
Processing of the multi-modal input MRI for a single subject, using the
multi-modal symmetric template, results in the generation of the feature
images. These feature images are used as input to the Stage 1 RF model
producing the initial RF probability map estimates. The Stage 1 voting
maps, the original feature images, and the Stage 2 RF model result in
the final voting maps which includes the WMH probability estimate. Note
that the RF models are constructed once from a set of training data
which are processed using the same feature-construction pipeline as the
single-subject input MRI.

\textbf{Figure 2:} Canonical views of the mutlivariate, bilaterally
symmetric template constructed from the MMRR data set (only shown are
the FLAIR, T1, and T2 modalities--- the components relevant for this
work). These images are important for asymmetry-based features.

\textbf{Figure 3:} Representation of Stage 1 feature images for subject
9. The FLAIR, T1-, and T2-weighted images are rigidly pre-aligned to the
space of the T1 image. The three modality images are then preprocessed
(N4 bias correction and adaptive denoising) followed by application of
standard ANTs brain extraction and \(n\)-tissue segmentation protocols
using the MMRR symmetric template and corresponding priors applied to
the T1 image. The feature images are then generated for voxelwise input
to the RF model which results in the voting maps illustrated on the
right. This gives a probabilistic classification of tissue type. Not
shown are the probability and voting images for the brain stem and
cerebellum.

\textbf{Figure 4:} Sample FLAIR acquisition image slices showing both
manual and random forest segmentations for both stages obtained during
the leave-one-out evaluation. Manual segmentations were performed by one
of the authors and provided the ground truth WMH labels for training the
random forest models.

\textbf{Figure 5:} Evaluation measures for both Stages of the
leave-one-out protocol of the described protocol in the Methods section:
(a) sensitivity, (b) positive predictive value, (c) \(F_1\) score, and
(d) relative volume difference. These quantitative assessments are given
for three quantile ranges spanning the range of the manually-derived
lesion volumes. Overall improvement in all three whole lesion-based
measures is seen as the second Stage RF model is applied for all three
quantile ranges. The relative volume difference corresponding to the
Stage 2 results tend to predict a decreased predicted volume over the
Stage 1 results.

\textbf{Figure 6:} Average \emph{MeanDecreaseAccuracy} plots generated
from the creation of all 24 random forest models for Stage 1 during the
leave-one-out evaluation. These plots are useful in providing a
quantitative assessment of the predictive importance of each feature.
Features are ranked in descending order of importance. The horizontal
error bars provide the \(95^{th}\) percentile and illustrate the
stability of the feature importance across the leave-one-out models. At
this initial stage only 31 feature images are used.

\textbf{Figure 7:} Average \emph{MeanDecreaseAccuracy} plots generated
from the creation of all 24 random forest models for Stage 2 during the
leave-one-out evaluation. These plots are useful in providing a
quantitative assessment of the predictive importance of each feature.
Features are ranked in descending order of importance. The horizontal
error bars provide the \(95^{th}\) percentile and illustrate the
stability of the feature importance across the leave-one-out models. We
augment the 31 feature images from the first stage by adding an
additional seven voting maps and 7 segmentation posteriors from
application of the Bayesian-based segmentation for a total of 45 images
for the second stage.

\textbf{Figure 8:} (a) FLAIR image slice illustrating WMHs which have
been manually delineated. The region around the WMHs is enlarged (b) in
the original FLAIR and the (c) contralateral FLAIR difference image.

\clearpage

\hypertarget{refs}{}

\end{document}
