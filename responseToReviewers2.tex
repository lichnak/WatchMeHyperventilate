\documentclass[12pt,]{article}
\usepackage{lmodern}
\usepackage{amssymb,amsmath}
\usepackage{ifxetex,ifluatex}
\usepackage{fixltx2e} % provides \textsubscript
\ifnum 0\ifxetex 1\fi\ifluatex 1\fi=0 % if pdftex
  \usepackage[T1]{fontenc}
  \usepackage[utf8]{inputenc}
\else % if luatex or xelatex
  \ifxetex
    \usepackage{mathspec}
    \usepackage{xltxtra,xunicode}
  \else
    \usepackage{fontspec}
  \fi
  \defaultfontfeatures{Mapping=tex-text,Scale=MatchLowercase}
  \newcommand{\euro}{€}
    \setmainfont{Georgia}
\fi
% use upquote if available, for straight quotes in verbatim environments
\IfFileExists{upquote.sty}{\usepackage{upquote}}{}
% use microtype if available
\IfFileExists{microtype.sty}{%
\usepackage{microtype}
\UseMicrotypeSet[protrusion]{basicmath} % disable protrusion for tt fonts
}{}
\usepackage[margin=1in]{geometry}
\ifxetex
  \usepackage[setpagesize=false, % page size defined by xetex
              unicode=false, % unicode breaks when used with xetex
              xetex]{hyperref}
\else
  \usepackage[unicode=true]{hyperref}
\fi
\hypersetup{breaklinks=true,
            bookmarks=true,
            pdfauthor={},
            pdftitle={},
            colorlinks=true,
            citecolor=blue,
            urlcolor=blue,
            linkcolor=magenta,
            pdfborder={0 0 0}}
\urlstyle{same}  % don't use monospace font for urls
\setlength{\parindent}{0pt}
\setlength{\parskip}{6pt plus 2pt minus 1pt}
\setlength{\emergencystretch}{3em}  % prevent overfull lines
\providecommand{\tightlist}{%
  \setlength{\itemsep}{0pt}\setlength{\parskip}{0pt}}
\setcounter{secnumdepth}{0}

%%% Use protect on footnotes to avoid problems with footnotes in titles
\let\rmarkdownfootnote\footnote%
\def\footnote{\protect\rmarkdownfootnote}

%%% Change title format to be more compact
\usepackage{titling}

% Create subtitle command for use in maketitle
\newcommand{\subtitle}[1]{
  \posttitle{
    \begin{center}\large#1\end{center}
    }
}

\setlength{\droptitle}{-2em}
  \title{}
  \pretitle{\vspace{\droptitle}}
  \posttitle{}
  \author{}
  \preauthor{}\postauthor{}
  \date{}
  \predate{}\postdate{}

\usepackage{booktabs}
\usepackage[final]{changes}
\usepackage[font={small},labelfont=bf,labelsep=colon]{caption}
\linespread{1.2}
\usepackage[compact]{titlesec}
\usepackage{enumitem}
\usepackage{tikz}
\def\checkmark{\tikz\fill[scale=0.4](0,.35) -- (.25,0) -- (1,.7) -- (.25,.15) -- cycle;}
\setlist{nolistsep}
\titlespacing{\section}{2pt}{*0}{*0}
\titlespacing{\subsection}{2pt}{*0}{*0}
\titlespacing{\subsubsection}{2pt}{*0}{*0}
\setlength{\parskip}{3pt}
\setremarkmarkup{(#2)}

% Redefines (sub)paragraphs to behave more like sections
\ifx\paragraph\undefined\else
\let\oldparagraph\paragraph
\renewcommand{\paragraph}[1]{\oldparagraph{#1}\mbox{}}
\fi
\ifx\subparagraph\undefined\else
\let\oldsubparagraph\subparagraph
\renewcommand{\subparagraph}[1]{\oldsubparagraph{#1}\mbox{}}
\fi

\begin{document}
\maketitle

\section{Response to reviewers (Round
2)}\label{response-to-reviewers-round-2}

\emph{\textcolor{blue}{We appreciate the additional time spent by the editors and
reviewers in evaluating the revised version of our manuscript.
Please see below for a point-by-point response to the minor issues raised
during this round.}}

\subsection{Reviewer 1}\label{reviewer-1}

The revised manuscript is significantly improved and much clearer. The
authors have addressed all my comments.

A couple of minor notes, Figures 7 and 8 seem to have the descriptions
swapped.

\begin{quote}
\emph{\textcolor{blue}{The figure captions had been incorrectly assigned during the
manuscript upload process which has since been corrected.}}
\end{quote}

Also, Table 1 shows Subject 2 with 0 lesions. Was there a specific
reason for including this subject in the study?

\begin{quote}
\emph{\textcolor{blue}{Yes.  We included all usable data from the source study including
the subject with 0 lesions.  In future work, we imagine using this tool in a typical
workflow which would include subjects without lesions and believed that an evaluation would
best be conducted if we included as much representative data as possible.
}}
\end{quote}

\subsection{Reviewer: 2}\label{reviewer-2}

After reviewing the revised manuscript I believe it can now be accepted.
Two minor issues: pp20-21: The formulas have errors i.e.~false positive
(FP) versus true negative (TN).

\begin{quote}
\emph{\textcolor{blue}{We assume that the reviewer is referring to the definitions for
sensitivity and PPV on pp. 20--21.  We consulted a number of sources and do not believe
the given equations are incorrect.  Perhaps the reviewer is referring to the previously
incomplete text given previous to the definition of sensitivity. This has since been
repaired in the revised manuscript.}}
\end{quote}

Also, the legends for figures 7 and 8 are swapped.

\begin{quote}
\emph{\textcolor{blue}{The figure captions had been incorrectly assigned during the
manuscript upload process which has since been corrected.}}
\end{quote}

\subsection{Reviewer: 3}\label{reviewer-3}

Regarding the ``imaging'' methodology, 3rd para, authors need to remove
the information about JRS, and instead just indicate `how' the work was
performed, since authorship already implies these people did the work.
For example, instead of ``The first author (J. R. S.) performed the
manual WMH tracings\ldots{}'' it probably should read something like
``Manual WMH tracings were performed\ldots{}.''.

\begin{quote}
\emph{\textcolor{blue}{We believe the current text provides important precision in specifying
how the manual tracings were performed, specifically in terms of qualifications and experience
of the person(s) involved.    Additionally, authorship does not necessarily imply performance of the manual tracings as
there are numerous labeled image data sets which have been made publicly available and used
by third-parties in the development of methodologies.}}
\end{quote}

\hypertarget{refs}{}

\end{document}
